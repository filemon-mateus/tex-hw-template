\documentclass{article}
\usepackage{preamble}

\begin{document}
\header{Filemon Mateus}{CS XXX}{Course Title Name}{Homework \# 1}{\today}
\begin{enumerate}
  [
    leftmargin={*},
    label={\arabic*.},
    font={\bf},
    labelsep={10pt},
    itemsep={20pt},
    ref={\arabic*}
  ]
  \item \label{qst:1}
    Here is a claim:
    \begin{claim}[Twin primes conjecture]\label{clm:1}
      There are infinitely many primes that are two apart.
    \end{claim}
    \begin{proof*}
      The proof is left as an exercise for the interested reader.
    \end{proof*}

  \item \label{qst:2}
    Here is a lemma:
    \begin{lemma}[Johnson-Lindenstrauss '84]\label{lma:1}
      A set of $n$ points in high dimensional Euclidean space can
      be mapped into an $O(\log n/\eps^2)$-dimensional Euclidean
      space such that the distance between any two points changes
      by only a factor of $(1 \pm \eps)$.
    \end{lemma}
    \begin{proof*}
      The proof is left as an exercise for the interested reader.
    \end{proof*}

  \item \label{qst:3}
    Here is a remark:
    \begin{remark}[Sexy primes conjecture]\label{rmk:1}
      There are infinitely many primes that are six apart.
    \end{remark}
    \begin{proof*}
      The proof is left as an exercise for the interested reader.
    \end{proof*}
  
  \item \label{qst:4}
    Here is a corollary:
    \begin{corollary}[Cousin primes conjecture]\label{cor:1}
      There are infinitely many primes that are four apart.
    \end{corollary}
    \begin{proof*}
      The proof is left as an exercise for the interested reader.
    \end{proof*}

  \item \label{qst:5}
    Here is a theorem:
    \begin{theorem}[Pythagorean theorem]\label{thm:1}
      For any right-triangle the square of the hypotenuse is equal
      to the sum of squares of the other two sides.
    \end{theorem}
    \begin{proof*}
      The proof is left as an exercise for the interested reader.
    \end{proof*}

  \item \label{qst:6}
    Here is a proposition based off \autoref{thm:1}:
    \begin{proposition}[Fermat's Last Theorem]\label{prop:1}
      For $a, b, c \in \N$, $a^n + b^n \neq c^n$ for any choices of $n
      > 2$.
    \end{proposition}
    \begin{proof*}
      I have a truly marvelous demonstration of this proposition that
      this margin is too narrow to contain.
    \end{proof*}

  \item \label{qst:7}
    Here is a definition:
    \begin{definition}
      Let $G = (V, E)$ be an undirected graph with edge-weights given
      by $w \colon E \rightarrow \R^+$. Assume that $w(e) \neq w(f)$
      whenever $e, f$ are distinct edges of $G$. We say that an edge
      is {\it treacherous} if it is the maximum weight edge of some cycle
      of $G$. On the other hand, an edge is {\it reliable} if it is not
      contained in any cycle of $G$.
    \end{definition}
\end{enumerate}
\end{document}
